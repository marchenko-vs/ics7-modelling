\documentclass{bmstu}

\usepackage{biblatex}
\usepackage{array}
\usepackage{amsmath}

\addbibresource{inc/biblio/sources.bib}

\begin{document}

\makereporttitle
    {Информатика, искусственный интеллект и системы управления} % Название факультета
    {Программное обеспечение ЭВМ и информационные технологии} % Название кафедры
    {лабораторной работе №~1} % Название работы (в дат. падеже)
    {Моделирование} % Название курса (необязательный аргумент)
    {Функции распределения и плотности вероятности некоторых случайных величин} % Тема работы
    {11} % Номер варианта (необязательный аргумент)
    {Марченко~В./ИУ7-73Б} % Номер группы/ФИО студента (если авторов несколько, их необходимо разделить запятой)
    {Рудаков~И.~В.} % ФИО преподавателя

{\centering \maketableofcontents}

\chapter{Теоретическая часть}

\section{Задачи на лабораторную работу}

Изучить два закона распределения случайной величины: равномерный и Эрланга. 
Разработать программу для построения графиков функции распределения и плотности вероятности равномерной случайной величины и случайной величины, распределенной по закону Эрланга.

\section{Равномерное распределение}

Говорят, что случайная величина имеет равномерное распределение на отрезке $[a,~b]$, где $a,~b \in \mathbb{R}$, если ее плотность вероятности $f_{X}(x)$ имеет вид:

\begin{equation}
f_{X}(x) = \begin{cases}
	\frac{1}{b - a},~x \in [a,~b], \\
	0,~x \notin [a,~b].
	\end{cases}
\end{equation}

Функция распределения имеет вид:

\begin{equation}
F_{X}(x) = \begin{cases}
	0,~x < a, \\
	\frac{x - a}{b - a},~a \leq x < b, \\
	1,~x \geq b.
	\end{cases}
\end{equation}

Обозначение равномерно распределенной случайной величины: \mbox{$X \sim U[a,~b]$}.

\section{Распределение Эрланга}

Распределение Эрланга представляет собой двухпараметрическое непрерывное распределение вероятностей при $x \in [0,~\infty)$. 
Два параметра: положительное целое число $k$ (т.~н. <<форма>>) и положительное действительное число $\lambda$ (т.~н. <<интенсивность>>). 
Иногда вместо параметра $\lambda$ используют т.~н. <<масштаб>> $\beta = \frac{1}{\lambda}$.

Плотность вероятности распределения Эрланга:

\begin{equation}
f_{X}(x;~k,~\lambda) = \frac{\lambda^{k} x^{k - 1} e^{-\lambda x}}{(k-1)!},~x,\lambda \geq 0.
\end{equation}

Если вместо $\lambda$ использовать $\beta$, то плотность вероятности будет иметь вид:

\begin{equation}
f_{X}(x;~k,~\beta) = \frac{x^{k - 1} e^{-\frac{x}{\beta}}}{\beta^{k} (k-1)!},~x,\beta \geq 0.
\end{equation}

Функция распределения случайной величины:

\begin{equation}
F_{X}(x;~k,~\lambda) = \frac{\gamma(k, \lambda x)}{\Gamma(k)} = \frac{\gamma(k, \lambda x)}{(k-1)!},
\end{equation}
где $\Gamma$ --- гамма-функция, а $\gamma$ --- нижняя неполная гамма-функция.

Обозначение случайной величины, распределенной по закону Эрланга: \mbox{$X \sim ${Erlang}$(k,~\lambda)$}.

Распределение Эрланга было разработано А.~К.~Эрлангом для определения количества телефонных звонков, которые могут быть совершены одновременно операторам коммутационных станций. 
Эта работа по организации телефонного трафика была расширена и теперь используется в системах массового обслуживания в целом. 
Распределение также используется в области случайных процессов.

\chapter{Примеры работы программы}

На рисунках~\ref{img:uniform-01}--\ref{img:uniform-03} показаны графики функции распределения и плотности вероятности равномерно распределенной случайной величины при различных параметрах $a$ и $b$.

\includeimage
    {uniform-01}
    {f}
    {H}
    {1\textwidth}
    {Функция распределения и плотность вероятности равномерно распределенной случайной величины при $a = -1$ и $b = 1$}
    
\includeimage
    {uniform-02}
    {f}
    {H}
    {1\textwidth}
    {Функция распределения и плотность вероятности равномерно распределенной случайной величины при $a = 5$ и $b = 7$}
    
\includeimage
    {uniform-03}
    {f}
    {H}
    {1\textwidth}
    {Функция распределения и плотность вероятности равномерно распределенной случайной величины при $a = 4.5$ и $b = 5.7$}
    
На рисунках~\ref{img:erlang-01}--\ref{img:erlang-04} показаны графики функции распределения и плотности вероятности случайной величины, распределенной по закону Эрланга при различных параметрах $k$ и $\lambda$.

\includeimage
    {erlang-01}
    {f}
    {H}
    {1\textwidth}
    {Функция распределения и плотность вероятности случайной величины, распределенной по закону Эрланга при $k = 1$ и $\lambda = 0.5$}
    
\includeimage
    {erlang-02}
    {f}
    {H}
    {1\textwidth}
    {Функция распределения и плотность вероятности случайной величины, распределенной по закону Эрланга при $k = 3$ и $\lambda = 0.5$}  
    
\includeimage
    {erlang-03}
    {f}
    {H}
    {1\textwidth}
    {Функция распределения и плотность вероятности случайной величины, распределенной по закону Эрланга при $k = 7$ и $\lambda = 2$}
    
\includeimage
    {erlang-04}
    {f}
    {H}
    {1\textwidth}
    {Функция распределения и плотность вероятности случайной величины, распределенной по закону Эрланга при $k = 9$ и $\lambda = 1$}

\end{document}
