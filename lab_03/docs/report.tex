\documentclass{bmstu}

\usepackage{biblatex}
\usepackage{array}
\usepackage{amsmath}

\addbibresource{inc/biblio/sources.bib}

\begin{document}

\makereporttitle
    {Информатика и системы управления}
    {Программное обеспечение ЭВМ и информационные технологии}
    {лабораторной работе №~3} % Название работы (в дат. падеже)
    {Моделирование} % Название курса (необязательный аргумент)
    {Генератор псевдослучайных чисел} % Тема работы
    {11} % Номер варианта (необязательный аргумент)
    {Марченко~В./ИУ7-73Б} % Номер группы/ФИО студента (если авторов несколько, их необходимо разделить запятой)
    {Рудаков~И.~В.} % ФИО преподавателя

{\centering \maketableofcontents}

\chapter{Теоретическая часть}

Для оценки случайности последовательности чисел обычно к ней применют около шести различных статистических критериев. 
Если последовательность им удовлетворяет, она считается случайной. 
Критерии делятся на два вида --- эмпирические и теоретические. 
Основные критерии проверки случайных чисел: $\chi^2$-критерий и критерий Колмогорова-Смирнова. 
В данной лабораторной работе для оценки последовательностей случайных чисел был выбран критерий <<хи-квадрат>>.

Критерий <<хи-квадрат>>, возможно, самый известный из всех статистических критериев. 
Он является основным методом, используемым в сочетании с другими критериями.

Предположим, что каждое наблюдение может принадлежать одной из $k$ категорий. 
Проводим $n$ независимых наблюдений. 
Это означает, что исход одного наблюдения абсолютно не влияет на исход других наблюдений. 
Пусть $p_s$ --- вероятность того, что каждое наблюдение относится к категории $s$, и пусть $Y_s$ --- число наблюдений, которые действительно относятся к категории $s$. 
Образуем статистику
\begin{equation}
	V = \sum_{s = 1}^{k} \frac{(Y_s + np_s)^2}{np_s}.
\end{equation}
Эту статистику можно записать в другом виде:
\begin{equation}
	V = \frac{1}{n} \sum_{s = 1}^{k} \frac{Y^2_s}{p_s} - n.
\end{equation}

Чтобы определить, является ли полученное значение $V$ приемлемым, воспользуемся таблицей процентных точек $\chi^2$-распределения. 
Используем строку таблицы с $v = k - 1$, т.~к. число степеней свободы на единицу меньше, чем число категорий. 
Сравниваем вычисленное $V$ со значением в таблице. 
Если $V$ меньше 1\%-й точки или больше 99\%-й, отбрасываем эти числа как недостаточно случайные. 
Если $V$ лежит между 1\%- и 5\%-й точками или между 95\%- и 99\%-й точками, то эти числа <<подозрительны>>. 
Если $V$ лежит между 5\%- и 10\%-й точками или между 90\%- и 95\%-й точками, то эти числа <<почти подозрительны>>.

Проверка по критерию $\chi^2$ часто производится три (и более) раза с разными данными. 
Если по крайней мере два из трех результатов оказываются подозрительными, то числа рассматриваются как недостаточно случайные.

\begin{table}[H]
\caption{Таблица процентных точек $\chi^2$-распределения для некоторых значений $v$}
\label{tabular:comparison}
\begin{tabular}{|p{1.25cm}|p{1.75cm}|p{1.75cm}|p{2cm}|p{2cm}|p{2cm}|p{2cm}|p{2cm}|}
\hline
$n - 1$ & $p = 1\%$ & $p = 5\%$ & $p = 25\%$ & $p = 50\%$ & $p = 75\%$ & $p = 95\%$ & $p = 99\%$
\tabularnewline
\hline
$v = 1$ & 0.00016 & 0.00393 & 0.1015 & 0.4549 & 1.323 & 3.841 & 6.635
\tabularnewline
\hline
$v = 2$ & 0.0201 & 0.1026 & 0.5754 & 1.386 & 2.773 & 5.991 & 9.21
\tabularnewline
\hline
$v = 9$ & 2.088 & 3.325 & 5.899 & 8.343 & 11.39 & 16.92 & 21.67
\tabularnewline
\hline
$v = 89$ & 60.93 & 68.25 & 79.68 & 88.33 & 97.6 & 112.02 & 122.94
\tabularnewline
\hline
$v = 899$ & 803.31 & 830.41 & 870.05 & 898.33 & 927.23 & 969.86 & 1000.57
\tabularnewline
\hline
\end{tabular}
\end{table}

Для генерации случайных чисел в данной лабораторной работе был выбран линейный конгруэнтный метод.

\chapter{Примеры работы программы}

На рисунках~\ref{img:view-01}--\ref{img:view-04} показаны примеры работы программы.

\includeimage
    {view-01}
    {f}
    {H}
    {1\textwidth}
    {Пример работы программы №~1}
    
\includeimage
    {view-02}
    {f}
    {H}
    {1\textwidth}
    {Пример работы программы №~2}
    
\includeimage
    {view-03}
    {f}
    {H}
    {1\textwidth}
    {Пример работы программы №~3}
    
\includeimage
    {view-04}
    {f}
    {H}
    {1\textwidth}
    {Пример работы программы №~4}

\end{document}
