\documentclass{bmstu}

\usepackage{biblatex}
\usepackage{array}
\usepackage{amsmath}

\addbibresource{inc/biblio/sources.bib}

\begin{document}

\makereporttitle
    {Информатика, искусственный интеллект и системы управления} % Название факультета
    {Программное обеспечение ЭВМ и информационные технологии} % Название кафедры
    {лабораторной работе №~2} % Название работы (в дат. падеже)
    {Моделирование} % Название курса (необязательный аргумент)
    {Цепь Маркова} % Тема работы
    {11} % Номер варианта (необязательный аргумент)
    {Марченко~В./ИУ7-73Б} % Номер группы/ФИО студента (если авторов несколько, их необходимо разделить запятой)
    {Рудаков~И.~В.} % ФИО преподавателя

{\centering \maketableofcontents}

\chapter{Теоретическая часть}

Случайный процесс, протекающий в системе $S$, называется марковским, если он обладает следующим свойством: для каждого момента времени $t_0$ вероятность любого состояния системы в будущем (при $t > t_0$) зависит только от ее состояния в настоящем (при $t = t_0$) и не зависит от того, когда и каким образом система пришла в это состояние. 
Вероятностью $i$-го состояния называется вероятность $p_i(t)$ того, что в момент $t$ система будет находиться в состоянии $S_i$. 
Для любого момента $t$ сумма вероятностей всех состояний равна единице.

Для решения поставленной задачи составляется система уравнений Колмогорова по следующему правилу: в левой части каждого уравнения стоит производная вероятности $i$-го состояния; в правой части --- сумма произведений вероятностей всех состояний (из которых идут стрелки в данное состояние), умноженная на интенсивности соответствующих потоков событий, минус суммарная интенсивность всех потоков, выводящих систему из данного состояния, умноженная на вероятность данного ($i$-го состояния).

Для получения предельных вероятностей, то есть вероятностей в стационарном режиме работы при $t \rightarrow \infty$, необходимо приравнять производные вероятностей к нулю. 
Таким образом получается система линейных уравнений. 
Для решения полученной системы необходимо добавить условие нормировки $p_0 + p_1 + ... + p_n = 1$.

На рисунке~\ref{img:chain} изображен граф переходов состояний.

\includeimage
    {chain}
    {f}
    {H}
    {0.5\textwidth}
    {Граф переходов состояний}
    
Здесь $\lambda _0 = \lambda _1 = \lambda _2 = \mu _1 = 1$, $\mu _2 = 2$, $\mu _3 = 3$. 
Тогда составленная для него система уравнений Колмогорова будет иметь вид:
\begin{equation}
 \begin{cases}
   \frac{dp_0(t)}{dt} = p_1(t) - p_0(t), \\
   \frac{dp_1(t)}{dt} = 2p_2(t) - p_1(t), \\
   \frac{dp_2(t)}{dt} = 3p_3(t) - 3p_2(t), \\
   \frac{dp_3(t)}{dt} = p_1(t) + p_2(t) - 3p_3(t), \\
   p_0(t) + p_1(t) + p_2(t) + p_3(t) = 1.
 \end{cases}
\end{equation}

Заменяем проивзодные на нули и решаем систему:
\begin{equation}
 \begin{cases}
   p_0 = \frac{1}{3}, \\
   p_1 = \frac{1}{3}, \\
   p_2 = \frac{1}{6}, \\
   p_3 = \frac{1}{6}.
 \end{cases}
\end{equation}

После того, как предельные вероятности будут найдены, нужно найти время пребывания системы в стационарном состоянии. 
Для этого необходимо с интервалом $\Delta t$ находить вероятность в момент времени $t + \Delta t$. 
Когда найденная вероятность будет равна соответствующей предельной с точностью до заданной погрешности,  можно завершить вычисления. 
Начальные значения для $p_i$ задаются. 
Например, $p_i = \frac{1}{n}$, где $n$ --- количество состояний системы.

\chapter{Примеры работы программы}

На рисунках~\ref{img:view-01}--\ref{img:view-03} показаны примеры работы программы при разном количестве состояний системы и при различных значениях интенсивностей.

\includeimage
    {view-01}
    {f}
    {H}
    {1\textwidth}
    {Пример работы программы при двух состояниях системы}
    
\includeimage
    {view-02}
    {f}
    {H}
    {1\textwidth}
    {Пример работы программы при пяти состояниях системы}
    
\includeimage
    {view-03}
    {f}
    {H}
    {1\textwidth}
    {Пример работы программы при десяти состояниях системы}

\end{document}
