\documentclass{bmstu}

\usepackage{biblatex}
\usepackage{array}
\usepackage{amsmath}

\addbibresource{inc/biblio/sources.bib}

\begin{document}

\makereporttitle
    {Информатика и системы управления}
    {Программное обеспечение ЭВМ и информационные технологии}
    {лабораторной работе №~4} % Название работы (в дат. падеже)
    {Моделирование} % Название курса (необязательный аргумент)
    {Моделирование системы массового обслуживания с одним генератором и обслуживающим аппаратом} % Тема работы
    {11} % Номер варианта (необязательный аргумент)
    {Марченко~В./ИУ7-73Б} % Номер группы/ФИО студента (если авторов несколько, их необходимо разделить запятой)
    {Рудаков~И.~В.} % ФИО преподавателя

{\centering \maketableofcontents}

\chapter{Теоретическая часть}

\section{Задачи на лабораторную работу}

Разработать программу для моделирования системы, состоящей из генератора, буферной памяти и обслуживающего аппарата. 
Генератор подает сообщения, распределенные по равномерному закону, они
приходят в память и выбираются на обработку по закону распределения Эрланга. 
Количество заявок конечно и задано. 
Предусмотреть случай, когда обработанная заявка возвращается обратно в очередь. 
Определить длину очереди, при которой не будет потерянных сообщений. 
Реализовать двумя способами: используя событийный и <<$\Delta t$>> принципы.

\section{Равномерное распределение}

Говорят, что случайная величина имеет равномерное распределение на отрезке $[a,~b]$, где $a,~b \in \mathbb{R}$, если ее плотность вероятности $f_{X}(x)$ имеет вид:
\begin{equation}
f_{X}(x) = \begin{cases}
	\frac{1}{b - a},~x \in [a,~b], \\
	0,~x \notin [a,~b].
	\end{cases}
\end{equation}

Функция распределения имеет вид:
\begin{equation}
F_{X}(x) = \begin{cases}
	0,~x < a, \\
	\frac{x - a}{b - a},~a \leq x < b, \\
	1,~x \geq b.
	\end{cases}
\end{equation}

Обозначение равномерно распределенной случайной величины: \mbox{$X \sim U[a,~b]$}.

Для генерации времени с помощью равномерного распределения используется следующая формула:
\begin{equation}
t = a + (b - a)R,
\end{equation}
где $R \sim U[0,~1]$.

\section{Распределение Эрланга}

Распределение Эрланга представляет собой двухпараметрическое непрерывное распределение вероятностей при $x \in [0,~\infty)$. 
Два параметра: положительное целое число $k$ (т.~н. <<форма>>) и положительное действительное число $\lambda$ (т.~н. <<интенсивность>>). 
Иногда вместо параметра $\lambda$ используют т.~н. <<масштаб>> $\beta = \frac{1}{\lambda}$.

Плотность вероятности распределения Эрланга:

\begin{equation}
f_{X}(x;~k,~\lambda) = \frac{\lambda^{k} x^{k - 1} e^{-\lambda x}}{(k-1)!},~x,\lambda \geq 0.
\end{equation}

Если вместо $\lambda$ использовать $\beta$, то плотность вероятности будет иметь вид:

\begin{equation}
f_{X}(x;~k,~\beta) = \frac{x^{k - 1} e^{-\frac{x}{\beta}}}{\beta^{k} (k-1)!},~x,\beta \geq 0.
\end{equation}

Функция распределения случайной величины:

\begin{equation}
F_{X}(x;~k,~\lambda) = \frac{\gamma(k, \lambda x)}{\Gamma(k)} = \frac{\gamma(k, \lambda x)}{(k-1)!},
\end{equation}
где $\Gamma$ --- гамма-функция, а $\gamma$ --- нижняя неполная гамма-функция.

Обозначение случайной величины, распределенной по закону Эрланга: \mbox{$X \sim ${Erlang}$(k,~\lambda)$}.

Для генерации времени с помощью распределения Эрланга используется следующая формула:
\begin{equation}
t = - \frac{1}{k \lambda} \sum_{i = 1}^{k} \ln(1 - R_i),
\end{equation}
где $R_i \sim U[0,~1]$.

\section{Принцип <<$\Delta t$>>}

Данный прицнип заключается в последовательном анализе состояний всех блоков системы в момент $t + \Delta t$ по заданному состоянию в момент $t$. 
При этом новое состояние блоков определяется в соответствии с их алгоритмическим описанием с учетом действующих случайных факторов. 
В результате этого анализа принимается решение о том, какие системные события должны имитироваться в данный момент времени. 
Основной недостаток: значительные затраты вычислительных ресурсов при малом значении $\Delta t$, а при больших значениях $\Delta t$ появляется вероятность пропуска события.

\section{Событийный принцип}

Характерное свойство модели системы обработки информации: состояние отдельных устройств изменяется в дискретные моменты времени, совпадающие с моментами поступления сообщения, окончания решения задачи, возникновения аварийных сигналов и т.~д. 
При использовании событийного принципа состояния всех блоков системы анализируется лишь в момент появления какого-либо события. 
Момент наступления следующего события определяется минимальным значением из списка будущих событий, представляющий собой совокупность моментов ближайшего изменения состояния каждого из блоков. 
Момент наступления следующего события определяется минимальным значением из списка событий.

\chapter{Примеры работы программы}

На рисунках~\ref{img:view-01}--\ref{img:view-04} показаны результаты работы программы при различных значениях параметров.

\includeimage
    {view-01}
    {f}
    {H}
    {.9\textwidth}
    {Результат работы программы №~1}
    
\includeimage
    {view-02}
    {f}
    {H}
    {.9\textwidth}
    {Результат работы программы №~2}
    
\includeimage
    {view-03}
    {f}
    {H}
    {.9\textwidth}
    {Результат работы программы №~3}
    
\includeimage
    {view-04}
    {f}
    {H}
    {.9\textwidth}
    {Результат работы программы №~4}

\end{document}
