\documentclass{bmstu}

\usepackage{biblatex}
\usepackage{array}
\usepackage{amsmath}

\addbibresource{inc/biblio/sources.bib}

\begin{document}

\makereporttitle
    {Информатика и системы управления}
    {Программное обеспечение ЭВМ и информационные технологии}
    {лабораторной работе №~5}
    {Моделирование}
    {Моделирование работы информационного центра}
    {}
    {Марченко~В./ИУ7-73Б} % Номер группы/ФИО студента
    {Рудаков~И.~В.}

{\centering \maketableofcontents}

\chapter{Теоретическая часть}

\section{Задачи на лабораторную работу}

В информационный центр приходят клиенты через интервал времени $10 \pm 2$ минуты. 
Если все три имеющихся оператора заняты, клиенту отказывают в обслуживании. 
Операторы имеют разную производительность и могут обеспечить обслуживание среднего запроса пользователя за $20 \pm 5$, $40 \pm 10$ и $40 \pm 20$ мин. 
Клиенты стремятся занять свободного оператора с максимальной производительностью. 
Полученные запросы сдаются в накопитель, откуда выбираются на обработку. 
На первый компьютер попадают запросы от 1-го и 2-го операторов, на второй --- от 3-го. 
Время обработки запроса первым компьютером --- 15 мин, вторым --- 30 мин. 
Промоделировать процесс обработки трехсот запросов. 
Определить вероятность отказа в обслуживании.

\includeimage
    {task}
    {f}
    {H}
    {.9\textwidth}
    {Схема информационного центра}

В процессе взаимодействия клиентов с информационным центром возможны два режима: режим нормального обслуживания (т.~е. клиент выбирает одного из свободных операторов, отдавая предпочтение тому, у которого меньше номер) и режим отказа в обслуживании, когда все операторы заняты.

Эндогенные переменные: время обработки задания $i$-м оператором и время решения этого задания $j$-м компьютером.

Экзогенные переменные: число обслуженных клиентов и число клиентов, получивших отказ.

Вероятность отказа можно вычислить по формуле $P = \frac{n_{d}}{n_{d} + n_{p}}$, где $n_{d}$ --- количество клиентов, получивших отказ, а $n_{p}$ --- количество обслуженных клиентов.

\chapter{Примеры работы программы}

На рисунках~\ref{img:view-01}--\ref{img:view-02} показаны результаты работы программы при различных значениях параметров.

\includeimage
    {view-01}
    {f}
    {H}
    {.8\textwidth}
    {Результат работы программы №~1}
    
\includeimage
    {view-02}
    {f}
    {H}
    {.8\textwidth}
    {Результат работы программы №~2}

\end{document}
